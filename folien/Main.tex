\documentclass[12pt,utf8]{beamer}
\setbeamercovered{transparent}
\usepackage[german]{babel}
\usepackage{color}
\usepackage{xcolor}
\usepackage{graphicx}
\usepackage{tikz}
%
% TU
% A Beamer theme with the colors of the TU Dortmund.
% derived from github.com/kdungs/beamertheme-vertex
%
%

%\usetheme{Szeged}

%\usepackage{fontspec}
%   \setmainfont[Ligatures=TeX]{Tex Gyre Adventor}
\usepackage{xcolor}


% Colors
\definecolor{FOSSAGgreen}{HTML}{0AA06F}
\definecolor{FOSSAGgreenLighter}{HTML}{0CC78A}
\definecolor{FOSSAGgreenDarker}{HTML}{097A56}
\definecolor{FOSSAGalert}{HTML}{A10000}
\definecolor{FOSSAGalertLighter}{HTML}{D03434}
\definecolor{FOSSAGexample}{HTML}{006496}
\definecolor{FOSSAGexampleLighter}{HTML}{8735B5}
\definecolor{PeP}{HTML}{565656}
\definecolor{ownDarkOr}{HTML}{0101DF}
\definecolor{ownDarkBlue}{HTML}{0080FF}
\definecolor{highlight}{HTML}{FF8A8B}

% Titlepage
%  Title
\setbeamerfont{title}{size=\Huge}
\setbeamercolor{title}{fg=white, bg=FOSSAGgreen}
%  Subtitle
\setbeamerfont{subtitle}{size=\large}
%  Author
%\setbeamerfont{author}{size=\large}
%  Institute
\setbeamerfont{institute}{series=\bfseries, size=\small}
\setbeamercolor{institute}{fg=PeP}

% Frametitle
\setbeamerfont{frametitle}{size=\Huge}
\setbeamercolor{frametitle}{fg=white, bg=FOSSAGgreen}

% Framesubtitle
\setbeamerfont{framesubtitle}{size=\large}
\setbeamercolor{framesubtitle}{fg=white, bg=FOSSAGgreenDarker}

% Structure
\setbeamerfont{structure}{size=\Large}
\setbeamercolor{structure}{fg=FOSSAGgreen}

% Blocks
\setbeamertemplate{blocks}[square]
\setbeamerfont{block title}{}
\setbeamercolor{block title}{fg=FOSSAGgreen, bg=white}
\setbeamercolor{block title alerted}{use=alerted text,fg=FOSSAGalert, bg=white}
\setbeamercolor{block title example}{use=example text,fg=FOSSAGexample, bg=white}

% Items
\setbeamertemplate{items}[circle]

% Other
\setbeamertemplate{navigation symbols}{} %no nav symbols

% Customizations


\setbeamertemplate{frametitle}{
	\ifx\insertframesubtitle\empty
	*\vspace{-0.5em}
	\begin{beamercolorbox}[wd=0.92\paperwidth, ht=0.9em, dp=0.2em, leftskip=0.04em, left]{frametitle}
		\hspace{.25em}\usebeamercolor{frametitle}\usebeamerfont{frametitle}\insertframetitle 
	\end{beamercolorbox}
	\else
	*\vspace{-0.5em}
	\begin{beamercolorbox}[wd=0.92\paperwidth, ht=0.9em, dp=0.2em, leftskip=0.04em, left]{frametitle}
		\hspace{.25em}\usebeamercolor{frametitle}\usebeamerfont{frametitle}\insertframetitle 
	\end{beamercolorbox}
	\begin{beamercolorbox}[wd=0.92\paperwidth, ht=0.7em, dp=0.2em, leftskip=0.04em, left]{framesubtitle}
		\hspace{.25em}\usebeamercolor{framesubtitle}\usebeamerfont{framesubtitle}\insertframesubtitle
	\end{beamercolorbox}
	\fi
}

\defbeamertemplate*{footline}{TU}{
	\leavevmode
	\begin{beamercolorbox}[wd=0\paperwidth, ht=0.1em, dp=1em, left, leftskip=1em]{}
		\textbf{\insertshorttitle}
	\end{beamercolorbox}
	\begin{beamercolorbox}[wd=.8\paperwidth, ht=0.1em, dp=1em, center, rightskip=4em]{}
		\insertshortauthor\quad(\insertshortinstitute)
	\end{beamercolorbox}
	\begin{beamercolorbox}[wd=.1\paperwidth, ht=0.1em, dp=1em, right, rightskip=-4em]{}
		\insertframenumber{} / \inserttotalframenumber
	\end{beamercolorbox}
}


\title{Linux Basics}
\subtitle{Das Terminal}
\author{Alexander Becker}
\institute[FOSS AG]{\textbf{F}ree and \textbf{O}pen \textbf{S}ource \textbf{S}oftware \textbf{AG}}

\date{\today}

\begin{document}
\begin{frame}
	\titlepage
\end{frame}

\begin{frame}
\frametitle{Inhaltsverzeichnis}
\begin{itemize}
	\item Einleitung
	\item Arbeiten mit Ordnern, Dateien und Archiven
	\item Systemverwaltung
\end{itemize}
\end{frame}

\begin{frame}
\frametitle{Unix Philosophie}
\begin{figure}
\includegraphics[scale=0.15]{res/tuX_tu.png}
\end{figure}
\begin{itemize}
	\item Philosophie besteht aus drei Punkten:
	\begin{enumerate}
		\item Schreibe Programme, die nur eine Sache tun und dies erfolgreich
		\item Schreibe Programme, die Kolaboration ermöglichen
		\item Schreibe Programme, die mit Text arbeiten, denn dies ist universell
	\end{enumerate}
\end{itemize}
\end{frame}

\begin{frame}
	\frametitle{\textcolor{FOSSAGalert}{l}i\textcolor{FOSSAGalert}{s}t}
	\framesubtitle{Ein kleines Licht in der Dunkelheit}
	\begin{itemize}
		\item Listet alle Ordner und Dateien in Verzeichnis auf
		\item Mögliche Optionen:
		\begin{itemize}[<+->]
			\item \texttt{-a}  (all) inkl. versteckter Verzeichnisse
			\item \texttt{-hl}  (human readable / long listing format) inkl. Rechte, Besitzer, Größe, etc.
		\end{itemize}
	\end{itemize}
\end{frame}

\begin{frame}
	\frametitle{\textcolor{FOSSAGalert}{c}hange \textcolor{FOSSAGalert}{d}irectory}
	\framesubtitle{Die kleine Form der Teleportation}
	\texttt{cd} ermöglicht das Navigieren durch das Dateisystem
	\begin{itemize}
		\item Wechsel in Parent-Directory mit \texttt{cd ..}
		\item Wechsel in Home-Directory mit \texttt{cd} ($\sim$)
	\end{itemize}
\end{frame}

\begin{frame}
\frametitle{\textcolor{FOSSAGalert}{c}o\textcolor{FOSSAGalert}{p}y $\&$ \textcolor{FOSSAGalert}{m}o\textcolor{FOSSAGalert}{v}e}
\framesubtitle{Die Kunst des Klonens und Umbenennens}
\begin{itemize}
	\item \texttt{mv PATH1 PATH2} (Verschieben von Dateien/Ordnern)
	\item \texttt{cp PATH1 PATH2} (Kopieren von Dateien)
	\begin{itemize}
		\item -r ermöglicht kopieren von Ordnern
	\end{itemize}
\end{itemize}
\end{frame}

\begin{frame}
\frametitle{\textcolor{FOSSAGalert}{m}a\textcolor{FOSSAGalert}{k}e \textcolor{FOSSAGalert}{dir}ectory}
\framesubtitle{Verzeichnisse erschaffen}
\texttt{mkdir} erstellt Ordner auf dem Rechner
\begin{itemize}
	\item \texttt{mkdir PATH}
	\item \texttt{-p} erstellt alle Ordner auf dem Pfad, die noch nicht existieren
\end{itemize}
\end{frame}

\begin{frame}
\frametitle{\textcolor{FOSSAGalert}{r}e\textcolor{FOSSAGalert}{m}ove}
\framesubtitle{Flutsch! Und weg!}
\begin{itemize}
	\item \texttt{rm [OPTION] FILE}
	\begin{itemize}[<+->]
		\item \texttt{-r} ermöglicht das Löschen von Ordnern
	\end{itemize}
\end{itemize}
\end{frame}

\begin{frame}
	\frametitle{find}
	\texttt{find} sucht Dateien im aktuellen Ordner und in Unterordnern
	\begin{itemize}
		\item \texttt{find -name FILE}
	\end{itemize}
\end{frame}

\begin{frame}
\frametitle{\textcolor{FOSSAGalert}{t}ape \textcolor{FOSSAGalert}{ar}chiver}
\framesubtitle{Archivieren}
Zum entpacken von Archiven wie \texttt{.tar}, \texttt{.tar.gz}, \texttt{.zip}, etc. 
\begin{itemize}
	\item \texttt{tar -czvf ARCHIVE\_PATH PATH\_FILES}\\(Verpacken)
	\item \texttt{tar -xzvf ARCHIVE\_PATH}\\(Entpacken)
\end{itemize}
\end{frame}

\begin{frame}
\frametitle{\textcolor{FOSSAGalert}{t}ape \textcolor{FOSSAGalert}{ar}chiver}
\framesubtitle{Archivieren}
\begin{itemize}[<+->]
	\item {\scriptsize \texttt{-c} (create) erzeugt neues Archiv}
	\item {\scriptsize \texttt{-x} (extract) extrahieren einer Datei}
	\item {\scriptsize \texttt{-v} (verbose) Fortschritt auflisten}
	\item {\scriptsize \texttt{-z} (gzip format) komprimieren als \texttt{.gz}, etc.}
	\item {\scriptsize \texttt{-f} erzeugt beim Entpacken einen Ordner mit Namen von Archiv}
\end{itemize}
\end{frame}

\begin{frame}
\frametitle{\textcolor{FOSSAGalert}{g}lobally search a \textcolor{FOSSAGalert}{r}egular \textcolor{FOSSAGalert}{e}xpression and \textcolor{FOSSAGalert}{p}rint}
\framesubtitle{Gonna catch em' Strings}
\texttt{grep} ermöglicht die Suche in einer Eingabe
\begin{itemize}
	\item \texttt{-i} ignoriert Groß- und Kleinschreibung bei der Suche
\end{itemize}
\end{frame}

\begin{frame}
\frametitle{man page}
\framesubtitle{Wenn man mal nicht weiter weiß}
Anleitungen (Manuals) zu den installierten Paketen lassen sich mit \texttt{man} betrachten.
\begin{itemize}
	\item man COMMAND
\end{itemize}
\end{frame}



\begin{frame}
\frametitle{Dateien betrachten und bearbeiten}
\begin{itemize}
	\item \texttt{cat FILE} - Ausgabe einer Datei
	\item \texttt{less FILE} - Betrachten einer Datei
	\item \texttt{nano FILE} - Bearbeiten einer Datei
\end{itemize}
\end{frame}

\begin{frame}
	\centering\includegraphics[scale=0.65]{res/IOE}
\end{frame}

\begin{frame}
	\Huge\centering{$|$~~~$>$}
\end{frame}

\begin{frame}
\Huge\centering{\&}
\end{frame}

\begin{frame}
\frametitle{\textcolor{FOSSAGalert}{h}isham \textcolor{FOSSAGalert}{t}able \textcolor{FOSSAGalert}{o}f \textcolor{FOSSAGalert}{p}rocesses}
\framesubtitle{interaktiver Prozessmanager}
\begin{itemize}
	\item \texttt{\textcolor{red}{t}able \textcolor{red}{o}f \textcolor{red}{p}rocesses} (bereits installtiert)
	\item \texttt{htop} (muss installiert werden)
	\begin{itemize}[<+->]
		\item {\scriptsize Übersicht über alle Prozesse und verbrauchte Ressourcen}
		\item {\scriptsize deutlich leichtere Bedienung und bessere Übersicht}
		\item {\scriptsize bietet mehr Interaktionen}
	\end{itemize}
\end{itemize}
\end{frame}

\begin{frame}
	\frametitle{\textcolor{FOSSAGalert}{s}uper \textcolor{FOSSAGalert}{u}ser \textcolor{FOSSAGalert}{do}}
	\framesubtitle{Mit großer Macht kommt große Verantwortung}
	\texttt{sudo} führt einen Befehl mit administrativer Berechtigung aus
	\begin{itemize}
		\item \texttt{sudo COMMAND}
	\end{itemize}
\end{frame}

\begin{frame}
	\frametitle{apt}
	Mit \texttt{apt} lassen sich Pakete installieren und deinstallieren
	\begin{itemize}
		\item \texttt{sudo apt install PACKAGE}
		\item \texttt{sudo apt update}
		\item \texttt{sudo apt upgrade / dist-upgrade}
		\item \texttt{sudo apt remove PACKAGE}
		\item \texttt{sudo apt search PACKAGE}
	\end{itemize}
\end{frame}

\begin{frame}
\frametitle{poweroff und reboot}
\framesubtitle{Wenn man mal das real life genießen will}
\begin{itemize}
	\item \texttt{poweroff} (Herunterfahren des Systems)
	\item \texttt{reboot} (Neustart des Systems)
\end{itemize}
\end{frame}

\begin{frame}
\frametitle{Übersicht}
%framesubtitle{Alle Befehle auf einen Blick}
\small
\begin{minipage}{0.48\linewidth}
	\begin{itemize}
		\item \texttt{ls} - Dateien auflisten
		\item \texttt{cp} - Datei kopieren
		\item \texttt{mv} - Datei verschieben
		\item \texttt{mkdir} - Ordner erstellen
		\item \texttt{rm} - Dateien löschen
		\item \texttt{find} - Dateien suchen
		\item \texttt{tar} - Archive verarbeiten
		\item \texttt{grep} - Text durchsuchen
		\item \texttt{man} - Anleitung anzeigen
	\end{itemize}
\end{minipage}
\begin{minipage}{0.48\linewidth}
	\begin{itemize}
		\item \texttt{less} - Datei betrachten
		\item \texttt{nano} - Texteditor
		\item \texttt{top/htop} - Prozessverwaltung
		\item \texttt{sudo} - Befehl als Root ausführen
		\item \texttt{apt} - Paketverwaltung
		\item \texttt{poweroff} - Herunterfahren
		\item \texttt{reboot} - Neustarten
	\end{itemize}
\end{minipage}

\end{frame}

\end{document}
