\documentclass[12pt,utf8]{beamer}

\usepackage[german]{babel}
\usepackage{xcolor}
\usepackage{graphicx}
\usepackage{tikz}

\title{Linux Terminal Basic}
\subtitle{BLABLABLA}

\author[J.-M. Lenk, C. Parnitzke, J. Schneider, Y. Bungers]{Jan-Marius Lenk, Christoph Parnitzke, Josef Schneider, Yannick Bungers}
\institute[FOSS AG]{Free and Open Source Software AG\\ Fakultät für Informatik}

\date{\today}

\begin{document}

\titlepage

\begin{frame}
\frametitle{Inhaltsverzeichnis}
\begin{itemize}
	\item Theorie
	\item Arbeiten mit Ordnern, Dateien und Archiven
	\item Prozesse
	\item Zusatz
\end{itemize}
\end{frame}

\section{Theorie}
\begin{frame}
\frametitle{Linux Philosophie}
\begin{figure}
\includegraphics[scale=0.3]{res/tux.png}
\end{figure}
\begin{itemize}
	\item Philosophie besteht aus drei Punkten:
	\begin{itemize}
		\item 1. Schreibe Programme, die nur eine Sache tun und dies erfolgreich (DOTADIW)
		\item 2. Schreibe Programme um zusammen zu Arbeiten
		\item 3. Schreibe Programme, die mit Text arbeiten, denn dies ist universell
	\end{itemize}
	\item Ist auch heute noch Kernaussage
\end{itemize}
\footnote{https://upload.wikimedia.org/wikipedia/commons/a/af/Tux.png}
\end{frame}

\begin{frame}
\frametitle{Terminal vs. Shell}
\begin{itemize}
	\item Terminal ist zeilenweise Eingabe von Befehlen
	\begin{itemize}
		\item[1)] TTY (teletype) sind Terminals, erreichbar via STRG+ALT+F[1-7]
		\item[2)] VTerm: virtueller Terminal in grafischer Oberfläche
		\item[3)] Shell: interpretiert Eingabe des Benutzers
	\end{itemize}
\end{itemize}
\end{frame}

\begin{frame}
\frametitle{apt-get vs. apt vs. aptitute}
\begin{itemize}
	\item Paket Management: apt Kurzform für apt-get und apt-cache
	\item wichtigsten Befehle:
	\begin{itemize}
		\item[1.] apt install [pkg...]
		\item[2.] apt update
		\item[3.] apt upgrade/dist-upgrade/full-upgrade
		\item[4.] apt remove [pkg...]
	\end{itemize}
	\item aptitude ist etwas grafischer und hat Zusatzfunktionen
\end{itemize}
\end{frame}

\begin{frame}
\frametitle{man page - Wenn man mal nicht weiter weiß}
\begin{itemize}
	\item Benutzeranleitung für das System
	\item Anleitung zu installierten Programmen/Tools
	\item Sektionen, die besonders interessant sind:
	\begin{itemize}
		\item NAME, SYNOPSIS, DESCRIPTION, OPTIONS (EXAMPLE und BUGS)
	\end{itemize}
	\item Wie werden Einträge gesucht?
	\begin{itemize}
		\item[1.] ausführbare Programme und Shell Kommandos
		\item[2.] Systemaufrufe (vom Kernel bereit gestellt)
		\item[3.] Bibliotheksaufrufe (innerhalb von Programm-Bibliotheken)
		\item[4.] spezielle Dateien (in /dev)
		\item[5.] Dateiformate und Konventionen (/etc/passwd)
		\item[6.] Spiele
		\item[7.] Sonstiges (Macropakete und Konventionen)
		\item[8.] System administrations Kommandos (nur Root)
		\item[9.] Kernel Routinen (kein Standard)
	\end{itemize}
\end{itemize}
\end{frame}

%autocompletion einfach erwähnen und als notiz auf Handout

\section{Arbeiten mit Ordnern, Dateien und Archiven}
\subsection{Ordner}
\begin{frame}
\frametitle{ls - kleines Licht in der Dunkelheit}
\begin{itemize}
	\item listet alle Ordner und Dateien in Verzeichnis auf
	\item aktuelles Verzeichnis ist dabei Default
	\item mögliche Argumente:
	\begin{itemize}[<+->]
		\item -a  (all) inkl. versteckter Verzeichnisse
		\item -l  (long listing format) inkl. Rechte, Besitzer, Größe, etc.
		\item -h  (human readable) Größen der Dateien leserlicher
		\item -r  (recursive) umgekehrte Reihenfolge
		\item -m  Namen durch Kommata getrennt
	\end{itemize}
\end{itemize}
\end{frame}

\begin{frame}
\frametitle{cd - die kleine Form der Teleportation}
cd bietet das Navigieren durch das Dateisystem
\begin{itemize}[<+->]
	\item Notationen:	
	\begin{itemize}[<+->]
		\item Wechsel in Parent-Directory mit ../
		\item Wechsel in Home-Directory mit cd   (ohne Pfad)
		\item Wechsel in Verzeichnis (Bsp.: Dokumente (bzw. Documents))
		\begin{itemize}[<+->]
			\item cd /home/[username]/Dokumente
			\item cd $\sim$/Dokumente   (Abkürzen von Home-Directory mit $\sim$)
			\item cd Dokumente    (relativ vom aktuellen Verzeichnis)
		\end{itemize}
	\end{itemize}
	\item Funfact: cd .  (wechselt in das aktuelle Verzeichnis)
	\begin{itemize}
		\item es tut nichts, aber dafür sehr gut
	\end{itemize}
\end{itemize}
\end{frame}

\begin{frame}
\frametitle{mkdir - Verzeichnisse erschaffen}
mkdir erstellt Ordner auf dem Rechner
\begin{itemize}[<+->]
	\item Syntax: mkdir [path]   (erzeugt Ordner mit Pfad [path])
	\item Beispiel:
	\begin{itemize}[<+->]
		\item mkdir neuer-ordner/    (relativ zum aktuellen Pfad)
		\item mkdir /tmp/neuer-ordner/tmp1/tmp2/
	\end{itemize}
	\item -p   erstellt Parent-Directory, falls es nicht existiert
	\item -mode=  Modus mit dem der Ordner erstellt wird (siehe chmod)
\end{itemize}
\end{frame}

\begin{frame}
\frametitle{chmod - Mein Ordner gehört mir}
chmod ändert die Zugriffsrechte von Dateien und Ordnern
\begin{itemize}[<+->]
	\item Syntax: chmod [mode] [path]
	\item Beispiel (/home/[username]/neuer-ordner/):
	\begin{itemize}[<+->]
		\item aktuelle Rechte: u=rwx g=r-x a=r-x
		\item chmod g=rwx $\sim$/neuer-ordner (wir geben der Gruppe Schreib-Recht)
		\item chmod a-r $\sim$/neuer-ordner (wir nehmen allen anderen das Lese-Recht)
		\item neue Rechte: u=rwx g=rwx a=--x
	\end{itemize}
\end{itemize}
\end{frame}

\begin{frame}
\frametitle{chown}
%TODO
\end{frame}

\begin{frame}
\frametitle{rmdir}
%TODO
\end{frame}

\begin{frame}
\frametitle{Input, Output und Error}
\begin{itemize}
	\item jedes Programm hat drei Datenströme:
	\begin{itemize}
		\item[1)] stdin (Standard Input [number: 0])
		\item[2)] stdout (Standard Output [number: 1])
		\item[3)] stderr (Standard Error Output [number: 2])
	\end{itemize}
	\item Bsp.: echo
	\begin{itemize}[<+->]
		\item echo gibt einen eingegebenen Text zurück
		\item Syntax: echo [text]
		\item [text] ist stdin und die Konsole ist stdout und stderr
	\end{itemize}
\end{itemize}
\end{frame}


\begin{frame}
\frametitle{Pipe, $<$ und $>$ }
Pipes helfen die Ein- und Ausgabe umzuleiten
\begin{itemize}[<+->]
	\item $<$ und $>$ laden und schreiben Dateien
	\begin{itemize}[<+->]
		\item $>$ Schreiben in eine Datei
		\item Bsp.: cal $>$ test.txt
		\item $<$ Laden aus einer Datei
		\item Bsp.: less $<$ test.txt
	\end{itemize}
	\item Pipe $\mid$
	\begin{itemize}[<+->]
		\item Ausgabe von Befehl $\to$ Eingabe anderer Befehl
		\item Bsp.: date $\mid$ less
	\end{itemize}
	\item fancy stuff:
	\begin{itemize}[<+->]
		\item Umleiten von stderr: 2$>$ statt $>$
		\item also durch Angabe von Stream Nummer
		\item Bsp.: Umleiten von stderr auf stdout
		\begin{itemize}
			\item ls -r $>$ output.txt 2$>\&$1 
		\end{itemize}
	\end{itemize}
\end{itemize}
\end{frame}

\begin{frame}
\frametitle{more - Less than everything}
more ist ein Filereader
\begin{itemize}
	\item Angucken von Datei ohne Editieren
	\item Arbeitet auf Standardausgabe
	\item Beherrscht alle dem System bekannten Codierungen
	\item Navigierung ist allerdings ein wenig hinderlich
	\item ist Grundlage zu less
\end{itemize}
\end{frame}

\begin{frame}
\frametitle{less - Even more than 'more'}
less ist intelligenter Nachfolger von more
\begin{itemize}
	\item unterstützt Scrolling
	\item Navigation:
	\begin{itemize}[<+->]
		\item Pfeil- und Bildlauftasten
		\item '/pattern' Durchsucht das Dokument nach angegebenen Muster
		\item '?pattern' ist wie /pattern nur rückwärts
		\item '$&$pattern' zeigt nur Zeilen, mit Muster
		\item wird pattern benutzt, dann zeigt 'n' das nächste Vorkommen
		\item ':n' ruft das nächste Dokument auf (wenn less mehrere bekommen hat)
	\end{itemize}
\end{itemize}
\end{frame}

\begin{frame}
\frametitle{tail und head}
%TODO
\end{frame}

\begin{frame}
\frametitle{cp und mv}
%TODO
\end{frame}

\end{document}